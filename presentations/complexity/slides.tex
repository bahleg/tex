\documentclass[10pt,pdf,utf8,russian,aspectratio=169]{beamer}
\usepackage[T2A]{fontenc}
\usepackage[english,russian]{babel}
%
% Choose how your presentation looks.
%
% For more themes, color themes and font themes, see:
% http://deic.uab.es/~iblanes/beamer_gallery/index_by_theme.html
%
\mode<presentation>
{
  \usetheme{Boadilla}      % or try Darmstadt, Madrid, Warsaw, ...
  \usecolortheme{seagull} % or try albatross, beaver, crane, ..

  \usefonttheme{structurebold}  % or try serif, structurebold, ...
  \setbeamertemplate{navigation symbols}{}
  \setbeamertemplate{caption}[numbered]
} 


\title[Сложность модели]{Сложность моделей глубокого обучения}
\author{Бахтеев Олег}
\institute{МФТИ}
\date{02.11.2016}

\begin{document}

\begin{frame}
  \titlepage
\end{frame}

% Uncomment these lines for an automatically generated outline.
\begin{frame}{План}
  \tableofcontents
\end{frame}

\section{Сложность модели}
\begin{frame}{Сложность модели}
Мотивация
\end{frame}


\begin{frame}{Minimum description length}
\end{frame}

\begin{frame}{MDL и Колмогоровская сложность}
\end{frame}

\begin{frame}{Refined MDL}
\end{frame}


\begin{frame}{Байесовый подход к сложности}
\end{frame}

\begin{frame}{Байесовый подход к сложности}
Порождение vs описание
\end{frame}

\begin{frame}{Кросс-валидация vs Байес}
\end{frame}

\section{Вариационная нижняя оценка}
\begin{frame}{Вариационная оценка}
Зачем нужна, что такое
\end{frame}

\begin{frame}{Пример: логистическая функция}
Копипсата работы Адуенко
\end{frame}

\begin{frame}{Получение вариацонной оценки}
Формула получения нижней оценки
\end{frame}

\begin{frame}{$D_\text{KL}$}

\end{frame}

\begin{frame}{Пример: нормальное распределение}
Из Бишопа
\end{frame}


\begin{frame}{Использование вариационной нижней оценки}
Зачем использовать? \\
Когда использовать?
\end{frame}

\section{Получение оценок для порождающих моделей}
\begin{frame}{Пример: автокодировщик}
Что такое?
\end{frame}

\begin{frame}{Автокодировщик как energy-based модель}
Интегралы и картинки из Bengio
\end{frame}

\begin{frame}{Вариационный автокодировщик}
Формулы
\end{frame}

\begin{frame}{Вариационный автокодировщик: правдоподобии модели}
Полная формула 
\end{frame}

\begin{frame}{Вариационный автокодировщик: правдоподобии модели}
Графики, примеры работы
\end{frame}

\section{Получение оценок для разделяющих моделей}
\begin{frame}{Разделяющие модели: правдоподобие}
аппроксимация нормальным распределением
\end{frame}

\begin{frame}{Градиентный спуск для оценки правдоподобия}
Иллюстрация
\end{frame}

\begin{frame}{Переобучение}
Иллюстрация
\end{frame}

\begin{frame}{Динамика Ланжевина}
иллюстрация
\end{frame}

\begin{frame}{Результаты}
\end{frame}

\end{document}

